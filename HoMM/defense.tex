\documentclass{beamer}
\setbeamerfont{subsection in toc}{size=\footnotesize}
%\setbeameroption{show notes on second screen=left} %enable for notes
\usepackage{graphicx}
\usepackage{xcolor}
\usepackage{listings}
\usepackage{hyperref}
\lstset{language=python,frame=single}
\usepackage{verbatim}
\usepackage[longnamesfirst]{natbib}
\usepackage{subcaption}
\usepackage{amsmath}
\usepackage{bm}
\usepackage{relsize}
\usepackage{appendixnumberbeamer}
\usepackage{xparse}
\usepackage{multimedia}
\usepackage{xcolor}
\usepackage[normalem]{ulem}
\usepackage{hyperref}
\usepackage{pdfpc-commands}
\usepackage{tikz}
\usetikzlibrary{matrix,backgrounds}
\usetikzlibrary{positioning}
\usetikzlibrary{shapes,arrows}
\usetikzlibrary{decorations.pathreplacing, calligraphy}

\captionsetup[subfigure]{labelformat=empty}

\tikzset{onslide/.code args={<#1>#2}{%
  \only<#1>{\pgfkeysalso{#2}} 
}}

\tikzstyle{block} = [rectangle, draw, thick, align=center, rounded corners]
\tikzstyle{boundingbox} = [thick, lightgray]
\tikzstyle{dashblock} = [rectangle, draw, thick, align=center, dashed]
\tikzstyle{conc} = [ellipse, draw, thick, dashed, align=center]
\tikzstyle{netnode} = [circle, draw, very thick, inner sep=0pt, minimum size=0.5cm]
\tikzstyle{relunode} = [rectangle, draw, very thick, inner sep=0pt, minimum size=0.5cm]
\tikzstyle{line} = [draw, very thick, -latex', -]
\tikzstyle{arrow} = [draw, ->, very thick]

\definecolor{bpurp}{HTML}{984ea3}
\definecolor{bblue}{HTML}{377eb8}
\definecolor{bgreen}{HTML}{4daf4a}
\definecolor{borange}{HTML}{ff7f00}
\definecolor{bred}{HTML}{e41a1c}

\usetheme[numbering=none]{metropolis}

\begin{document}

\title{Toward understanding human adaptibility with deep learning models}
\author{Andrew Lampinen}
\date{May 12th, 2020}
\frame{\titlepage}

\begin{frame}<1-2>[label=pokermotivation]
\frametitle<1-2>{Human cognition is flexible -- we can adapt on our first try}
\frametitle<3-4>{How do humans adapt?}
\frametitle<5>{Switching to opposite rewards is challenging}
\centering
\only<1,5>{
\includegraphics[width=\textwidth]{figures/poker.png}
}
\only<2>{
\includegraphics[width=\textwidth]{figures/old_people_looking_0.png}
}
\only<3>{
\includegraphics[width=\textwidth]{figures/poker_adaptation_0.png}
}
\only<4>{
\includegraphics[width=\textwidth]{figures/poker_adaptation_1.png}
}
\end{frame}
%% Don't forget to say starting point!!!

\begin{frame}{While deep learning is effective at individual tasks...}
\centering
\begin{tikzpicture}[remember picture]
\node at (-1.5, 2) {\includegraphics[width=0.66\textwidth]{figures/imagenet.jpg}};
\node at (-3, -1.5) {\includegraphics[width=0.3\textwidth]{figures/dqn_atari_human.jpg}};
\node at (3.8, 2) {\includegraphics[width=0.3\textwidth]{figures/alphago.jpg}};
\node at (2.25, -1.5) {\includegraphics[width=0.66\textwidth]{figures/poker_superhuman.png}};
\end{tikzpicture}
\end{frame}

\begin{frame}{... it is often criticized for its inflexibility}
\includegraphics[width=\textwidth]{figures/deep_learning_flexibility_critiques.png}
{\scriptsize (Lake, Ullman, Tenenbaum, Gershman, 2016; Marcus, 2018; Hao, 2020)}
\end{frame}

\begin{frame}[standout]
How can we build deep-learning models with more human-like flexibility?
\end{frame}

\againframe<3-4>{pokermotivation}

\begin{frame}[standout]
Goal: a deep-learning framework that can adapt to task transformations. 
\end{frame}



\begin{frame}{Prior work on flexible deep learning models}
Lots of prior work. In particular: 
\begin{itemize}[<+(1)->]
    \item Lots of work on meta-learning (learning to learn).
    \item Work on zero-shot task performance from language.
    \item Model-based methods. 
\end{itemize}
\uncover<5>{
I'll discuss the relationship between my work and these as I go along.
}
\end{frame}

\section{Meta-mapping}

\begin{frame}{Tasks as functions/mappings}
\begin{columns}
\begin{column}{0.5\textwidth}
It's useful to think of tasks or behaviors as functions mapping input to output:
\begin{itemize}
    \item Poker hand \(\rightarrow\) bet
    \item Chess position \(\rightarrow\) move
    \item Object \(\rightarrow\) classification
\end{itemize}
Standard deep learning tries to infer this mapping from lots of examples. 
\end{column}

\begin{column}{0.5\textwidth}
\includegraphics[width=\textwidth]{figures/poker_function.png}
\end{column}
\end{columns}
\end{frame}

\begin{frame}[standout]
Tasks \(\bm =\) mappings from inputs to outputs. 
\end{frame}

%% TODO: Maybe cut
\begin{frame}{Meta-learning}
\begin{columns}
\begin{column}{0.5\textwidth}

\begin{itemize}[<+->]
    \item Lots of recent research on meta-learning -- learning to learn.
    \item If you have learned a lot of card games, you should be able to pick up a new one from a few examples.
    \item Then apply this knowledge to play the game with probe hands you weren't taught. 
    \item But, again, how could we play a variation of the game without examples? 
\end{itemize}
\end{column}

\begin{column}{0.5\textwidth}
\includegraphics[width=\textwidth]{figures/poker_function_2.png}
\end{column}
\end{columns}
\end{frame}

\begin{frame}{Altering the task = adapting the mapping}
\only<1>{
\includegraphics[width=\textwidth]{figures/poker_adaptation_1.png}
}
\only<2>{
\includegraphics[width=\textwidth]{figures/poker_adaptation_mappings.png}
}
\end{frame}

\begin{frame}[standout]
We want to model how we can transform the task mapping.
\end{frame}

\begin{frame}{Meta-mappings}
How do we adapt a task-mapping?
\begin{itemize}
\item We propose \textbf{meta-mappings}, higher-order mappings which transform task mappings into other task mappings.
\end{itemize}
\includegraphics[width=\textwidth]{figures/meta_mapping_poker.png}
\end{frame}

\begin{frame}[standout]
Transforming the task mapping \(=\) adapting. \\[1em]
Meta-mappings are higher-order mappings that transform task mappings.
\end{frame}

\begin{frame}<-3>[label=basic_metamapping_analogy]
\frametitle{Meta-mappings are analogous to basic tasks}
\begin{columns}
\begin{column}{0.5\textwidth}
\vspace{2em}
\includegraphics[width=\textwidth]{figures/poker_function.png}
\end{column}
\begin{column}{0.5\textwidth}
\vspace{2em}
\includegraphics[width=\textwidth]{figures/try_to_lose_function.png}
\end{column}
\end{columns}
\only<-3>{
\begin{itemize}[<+->]
\item Both are just functions (they just have different types of inputs and outputs).
\item This means we can apply all the usual deep learning tricks to meta-mappings.
\item ... Assuming we have a way to represent the basic tasks for input to (and output from) the meta-mappings. 
\end{itemize}
}
\end{frame}

\begin{frame}[standout]
There is a functional analogy between basic tasks and meta-mappings. \\[1em]
But how can we use it?
\end{frame}

\section{An architecture for representing and transforming tasks}

\begin{frame}{Architecture preview}
Our goal is to implement:
\begin{itemize}
\item A way of representing tasks as mappings.
\item A way of representing those tasks/mappings as vectors.
\item A way of learning to transform those task-vectors to perform meta-mappings.
\end{itemize}
This will allow the architecture to perform new tasks without data, by their relationship to prior tasks.
\end{frame}

\begin{frame}{Task-specific computations are low-dimensional and abstract}
\begin{tikzpicture}[auto]
\node at (-4, 0) (image) {\includegraphics[width=2cm]{figures/straight_flush_hand.jpg}};


%% input

\node[netnode] at (-2.25, -1.5) (i00) {};
\node[netnode] at (-2.25, -0.75) (i01) {};
\node[netnode] at (-2.25, 0) (i02) {};
\node[netnode] at (-2.25, 0.75) (i03) {};
\node[netnode] at (-2.25, 1.5) (i04) {};

\path [line] ([xshift=-3]image.east) to (i00);
\path [line] ([xshift=-3]image.east) to (i01);
\path [line] ([xshift=-3]image.east) to (i02);
\path [line] ([xshift=-3]image.east) to (i03);
\path [line] ([xshift=-3]image.east) to (i04);

\node[netnode] at (-1, -1.5) (i10) {};
\node[netnode] at (-1, -0.75) (i11) {};
\node[netnode] at (-1, 0) (i12) {};
\node[netnode] at (-1, 0.75) (i13) {};
\node[netnode] at (-1, 1.5) (i14) {};

\path [line] (i00) to (i10);
\path [line] (i00) to (i11);
\path [line] (i00) to (i12);
\path [line] (i00) to (i13);
\path [line] (i00) to (i14);
\path [line] (i01) to (i10);
\path [line] (i01) to (i11);
\path [line] (i01) to (i12);
\path [line] (i01) to (i13);
\path [line] (i01) to (i14);
\path [line] (i02) to (i10);
\path [line] (i02) to (i11);
\path [line] (i02) to (i12);
\path [line] (i02) to (i13);
\path [line] (i02) to (i14);
\path [line] (i03) to (i10);
\path [line] (i03) to (i11);
\path [line] (i03) to (i12);
\path [line] (i03) to (i13);
\path [line] (i03) to (i14);
\path [line] (i04) to (i10);
\path [line] (i04) to (i11);
\path [line] (i04) to (i12);
\path [line] (i04) to (i13);
\path [line] (i04) to (i14);

%% task specific
\node[netnode] at (0.25, -0.375) (t00) {};
\node[netnode] at (0.25, 0.375) (t01) {};

\path [line] (i10) to (t00);
\path [line] (i10) to (t01);
\path [line] (i11) to (t00);
\path [line] (i11) to (t01);
\path [line] (i12) to (t00);
\path [line] (i12) to (t01);
\path [line] (i13) to (t00);
\path [line] (i13) to (t01);
\path [line] (i14) to (t00);
\path [line] (i14) to (t01);

\node[netnode] at (1.5, -0.375) (t10) {};
\node[netnode] at (1.5, 0.375) (t11) {};

\path [line] (t00) to (t10);
\path [line] (t00) to (t11);
\path [line] (t01) to (t10);
\path [line] (t01) to (t11);

%% output

\node[netnode] at (2.75, -1.5) (o00) {};
\node[netnode] at (2.75, -0.75) (o01) {};
\node[netnode] at (2.75, 0) (o02) {};
\node[netnode] at (2.75, 0.75) (o03) {};
\node[netnode] at (2.75, 1.5) (o04) {};

\path [line] (t10) to (o00);
\path [line] (t11) to (o00);
\path [line] (t10) to (o01);
\path [line] (t11) to (o01);
\path [line] (t10) to (o02);
\path [line] (t11) to (o02);
\path [line] (t10) to (o03);
\path [line] (t11) to (o03);
\path [line] (t10) to (o04);
\path [line] (t11) to (o04);

\node[netnode] at (4, -1.5) (o10) {};
\node[netnode] at (4, -0.75) (o11) {};
\node[netnode] at (4, 0) (o12) {};
\node[netnode] at (4, 0.75) (o13) {};
\node[netnode] at (4, 1.5) (o14) {};

\path [line] (o00) to (o10);
\path [line] (o00) to (o11);
\path [line] (o00) to (o12);
\path [line] (o00) to (o13);
\path [line] (o00) to (o14);
\path [line] (o01) to (o10);
\path [line] (o01) to (o11);
\path [line] (o01) to (o12);
\path [line] (o01) to (o13);
\path [line] (o01) to (o14);
\path [line] (o02) to (o10);
\path [line] (o02) to (o11);
\path [line] (o02) to (o12);
\path [line] (o02) to (o13);
\path [line] (o02) to (o14);
\path [line] (o03) to (o10);
\path [line] (o03) to (o11);
\path [line] (o03) to (o12);
\path [line] (o03) to (o13);
\path [line] (o03) to (o14);
\path [line] (o04) to (o10);
\path [line] (o04) to (o11);
\path [line] (o04) to (o12);
\path [line] (o04) to (o13);
\path [line] (o04) to (o14);

\node at (5.25, 0) (output) {\textbf{\$\$\$}};
\path [line] (o10) to (output.west);
\path [line] (o11) to (output.west);
\path [line] (o12) to (output.west);
\path [line] (o13) to (output.west);
\path [line] (o14) to (output.west);

%% overlays
\uncover<2->{
\draw[fill=black, opacity=0.2] (-2.9, -3.5) rectangle (-0.25, 2);
\draw[fill=black, opacity=0.2] (2, -3.5) rectangle (4.5, 2);
\draw[fill=bblue, opacity=0.4] (-0.25, -3.5) rectangle (2, 2);
}

%% anotations

\node at (-4, -3) {\textbf{Input}};
\node[text width=2.5cm, align=center] at (-1.625, -3) {\textbf{Shared ``perception''}};
\node[text width=2cm, align=center] at (0.875, -3) {\textbf{Task specific}};
\node[text width=2.5cm, align=center] at (3.375, -3) {\textbf{Shared ``action''}};
\node at (5.25, -3) {\textbf{Output}};


\end{tikzpicture}
\end{frame}

\begin{frame}<-2>[label=homm_basic]
\frametitle{A multi-task architecture}
\centering
\begin{columns}
\begin{column}{\dimexpr\paperwidth-1em}
\centering
\resizebox{\textwidth}{!}{%
\begin{tikzpicture}[auto, every node/.style={execute at begin node=\setlength{\baselineskip}{1.1em}}]
%% a) constructing
\begin{scope}[shift={(0.4, 0)}]
\draw[boundingbox, draw=gray, fill=white] (-8.5, -2.5) rectangle (-2, 2.5);

%% from language
\node[gray] at (-7, 2.1) {Instructions};
\node at (-7, 1.25) (language) {``Play poker.''};

\node[gray, text width=2cm, align=center] at (-4.1, 2.1) {Language network};
\node[block] at (-4.1, 1.25) (languagenet) {\(\mathcal{L}\)};
\path[arrow] (language.east) -- ([xshift=-3]languagenet.west);

\node[bpurp, text width=2cm] at (-2, 1.25) (languagetaskrep) {\(z_{poker}\)};
\path[arrow] ([xshift=3]languagenet.east) -- (languagetaskrep.west);


%% from examples

\node[gray, text width=2.5cm, align=center] at (-7, -0.2) {Task examples (encoded)};
\node at (-7, -1.25) (examples) {
\(\left\{
\begin{matrix}
({\color{bgreen}z_{hand_{1}}}, {\color{bgreen}z_{bet_{1}}})\\
$\vdots$
\end{matrix}\right\}\)};

\node[gray, text width=2cm, align=center] at (-4.1, -0.4) {Example network};
\node[block] at (-4.1, -1.25) (examplenet) {\(\mathcal{E}\)};
\path[arrow] (examples.east) -- ([xshift=-3]examplenet.west);

\node[bpurp, text width=2cm] at (-2, -1.25) (examplestaskrep) {\(z_{poker}\)};
\path[arrow] ([xshift=3]examplenet.east) -- (examplestaskrep.west);
\end{scope}

%% b) performing
\uncover<2->{
\draw[boundingbox, draw=gray, fill=white] (-1.5, -2.5) rectangle (8, 2.5);
\node[bpurp] at (-0.75, 1.25) (taskrep) {\(z_{poker}\)};

\node[gray, text width=2cm, align=center] at (0.75, 2.1) {Hyper network};
\node[block] at (0.75, 1.25) (hypernet) {\(\mathcal{H}\)};
\path[arrow] (taskrep.east) -- ([xshift=-3]hypernet.west);

\node[text width=0.5cm] at (-0.85, -1.25) (inputs) {\includegraphics[width=0.5cm]{figures/2_of_spades.png}\\\includegraphics[width=0.5cm]{figures/4_of_hearts.png}};

\node[gray, text width=2cm, align=center] at (0.6, -2) {Perception network};
\node[block] at (0.6, -1.25) (perceptionnet) {\(\mathcal{P}\)};
\path[arrow] (inputs.east) -- ([xshift=-3]perceptionnet.west);

\node[bgreen] at (2.05, -1.25) (handrep) {\(z_{hand}\)};
\path[arrow] ([xshift=3]perceptionnet.east) -- (handrep.west);

\node[bblue, block, dashed] at (3.5, -1.25) (tasknet) {\(\mathcal{T}\)};
\node[bblue, text width=1.5cm, align=center] at (3.5, -2) {Task network};
\path[arrow] (handrep.east) -- ([xshift=-3]tasknet.west);
\path[arrow, out=0, in=90] ([xshift=3]hypernet.east) to ([yshift=3]tasknet.north);


\node[bgreen] at (4.85, -1.25) (betrep) {\(z_{bet}\)};
\path[arrow] ([xshift=3]tasknet.east) -- (betrep.west);

\node[gray, text width=2cm, align=center] at (6.2, -2) {Action network};
\node[block] at (6.2, -1.25) (actionnet) {\(\mathcal{A}\)};
\path[arrow] (betrep.east) -- ([xshift=-3]actionnet.west);

\node at (7.4, -1.25) (output) {\bf \$};
\path[arrow] ([xshift=3]actionnet.east) -- (output.west);
}

%%% d subpanel) task network
%
%\draw[boundingbox, draw=gray, fill=white] (4.25, 0) rectangle (8.5, 2.5);
%\node[gray] at (6.35, 2.25) {\(\mathcal{H}\) adapts all weights in \(\mathcal{T}\)};
%
%\node at (4.45, 1.45) (pseudohyper) {};
%
%\node[bblue] at (5.25, 1.65) (Wprime) {\scriptsize \(W'\), \(b'\) (from \(\mathcal{H}\))};
%\node[bblue] at (6.05, 0.25) (W0) {\scriptsize Default \(W^{0}\), \(b^{0}\)};
%\node[netnode, bblue, thick] at (4.75, 0.5) (tnode1) {I};
%
%\node[netnode, bblue, thick] at (7.35, 0.5) (tnode2) {O};
%\node[bblue] at (7.35, 1.4) {\scriptsize\(O\!=\!(W^{0}\!+\!W') I\)};
%\node[bblue] at (7.4, 1.1) {\scriptsize\(+(b^{0}\!+\!b')\)};
%
%\path[arrow, bblue, out=0, in=90] (pseudohyper) to ([yshift=10]W0);
%\path[arrow, bblue, dashed] (tnode1) to (tnode2);
%
%%% connecting to subpanel
%
%\path[draw, gray, thick] (3.75, -0.95) to (4.25, 2.5);
%\path[draw, gray, thick] (3.75, -0.95) to (8.5, 0);

\uncover<4->{
\draw [red, line width=0.25em] (-0.75, 1.25) circle (0.5);
}
\end{tikzpicture}}
\end{column}
\end{columns}
\end{frame}

\begin{frame}[standout]
We use an architecture that generates task representations, which it uses to parameterize (part of) a network which performs the task. % shorten this
\end{frame}

\againframe<4>{homm_basic}

\begin{frame}<1>[label=homm_arch]
\frametitle<1>{Meta-mapping as transforming function representations}
\frametitle<2-3>{So we can use the same approach and networks!}
\frametitle<4>{Architecture reminder}
\frametitle<5>{Language-based approach}
\frametitle<6>{Language-based meta-mapping}
%\frametitle<4>{Language-based meta-mapping}
\begin{columns}
\begin{column}{\dimexpr\paperwidth-10pt}
\centering
\only<3->{
\resizebox{\dimexpr\textwidth-1pt}{!}{%
\begin{tikzpicture}[auto, every node/.style={execute at begin node=\setlength{\baselineskip}{1.1em}}]
\begin{scope}[shift={(0.4, 0)}]
%% a) constructing
\draw[boundingbox, draw=gray, fill=white] (-8.5, -2.5) rectangle (-2, 2.5);

%% from language
\only<-3, 5-6>{
\node[gray] at (-7, 2.1) {Instructions};
\node at (-7, 1.25) (language) {``Play poker.''};

\node[gray, text width=2cm, align=center] at (-4.1, 2.1) {Language network};
\node[block] at (-4.1, 1.25) (languagenet) {\(\mathcal{L}\)}; 
\path[arrow] (language.east) -- ([xshift=-3]languagenet.west);

\node[bpurp, text width=2cm] at (-2, 1.25) (languagetaskrep) {\(z_{poker}\)}; 
\path[arrow] ([xshift=3]languagenet.east) -- (languagetaskrep.west);
}


%% from examples
\uncover<-4>{

%\only<4>{
%\begin{scope}[shift={(0, 2.5)}]
%}

\node[gray, text width=2.5cm, align=center] at (-7, -0.2) {Task examples (encoded)};
\node at (-7, -1.25) (examples) {
\(\left\{
\begin{matrix}
({\color{bgreen}z_{hand_{1}}}, {\color{bgreen}z_{bet_{1}}})\\
$\vdots$
\end{matrix}\right\}\)};

%\only<-3, 5-6>{
\node[gray, text width=2cm, align=center] at (-4.1, -0.4) {Example network};
%}
\node[block] at (-4.1, -1.25) (examplenet) {\(\mathcal{E}\)}; 
\path[arrow] (examples.east) -- ([xshift=-3]examplenet.west);

\node[bpurp, text width=2cm] at (-2, -1.25) (examplestaskrep) {\(z_{poker}\)}; 
\path[arrow] ([xshift=3]examplenet.east) -- (examplestaskrep.west);
%\only<4>{
%\end{scope}
%}
}

\end{scope}
%% b) performing 
\draw[boundingbox, draw=gray, fill=white] (-1.5, -2.5) rectangle (8.5, 2.5);
\node[bpurp] at (-0.75, 1.25) (taskrep) {\(z_{poker}\)};

\node[gray, text width=2cm, align=center] at (0.75, 2.1) {Hyper network};
\node[block] at (0.75, 1.25) (hypernet) {\(\mathcal{H}\)}; 
\path[arrow] (taskrep.east) -- ([xshift=-3]hypernet.west);

\node[text width=0.5cm] at (-0.85, -1.25) (inputs) {\includegraphics[width=0.5cm]{figures/2_of_spades.png}\\\includegraphics[width=0.5cm]{figures/4_of_hearts.png}};

\node[gray, text width=2cm, align=center] at (0.6, -2) {Perception network};
\node[block] at (0.6, -1.25) (perceptionnet) {\(\mathcal{P}\)}; 
\path[arrow] (inputs.east) -- ([xshift=-3]perceptionnet.west);

\node[bgreen] at (2.05, -1.25) (handrep) {\(z_{hand}\)};
\path[arrow] ([xshift=3]perceptionnet.east) -- (handrep.west);

\node[bblue, block, dashed] at (3.5, -1.25) (tasknet) {\(\mathcal{T}\)}; 
\node[bblue, text width=1.5cm, align=center] at (3.5, -2) {Task network};
\path[arrow] (handrep.east) -- ([xshift=-3]tasknet.west);
\path[arrow, out=0, in=90] ([xshift=3]hypernet.east) to ([yshift=3]tasknet.north);


\node[bgreen] at (4.85, -1.25) (betrep) {\(z_{bet}\)};
\path[arrow] ([xshift=3]tasknet.east) -- (betrep.west);

\node[gray, text width=2cm, align=center] at (6.2, -2) {Action network};
\node[block] at (6.2, -1.25) (actionnet) {\(\mathcal{A}\)}; 
\path[arrow] (betrep.east) -- ([xshift=-3]actionnet.west);

\node at (7.4, -1.25) (output) {\bf \$};
\path[arrow] ([xshift=3]actionnet.east) -- (output.west);
\end{tikzpicture}}}\\%
\resizebox{\textwidth}{!}{%
\begin{tikzpicture}[auto, every node/.style={execute at begin node=\setlength{\baselineskip}{1.1em}}]
\setlength{\baselineskip}{1.2em}
\only<2-4,6>{
\begin{scope}[shift={(0.4, 0)}]
%% a) constructing
\draw[boundingbox, draw=gray, fill=white] (-8.5, -2.5) rectangle (-2, 2.5);

%% from language
\only<-3, 5->{
\node[gray] at (-6.7, 2.1) {Instructions};
\node at (-6.7, 1.25) (language) {``Try to lose.''};

\node[gray, text width=2cm, align=center] at (-4.1, 2.1) {Language network};
\node[block] at (-4.1, 1.25) (languagenet) {\(\mathcal{L}\)}; 
\path[arrow] (language.east) -- ([xshift=-3]languagenet.west);

\node[borange, text width=2cm] at (-2, 1.25) (languagetaskrep) {\(z_{meta}\)}; 
\path[arrow] ([xshift=3]languagenet.east) -- (languagetaskrep.west);
}


%% from examples

\only<-5>{
\node[gray, text width=3.5cm, align=center] at (-6.7, -0.2) {Mapping examples (input/output tasks)};
\node at (-6.7, -1.25) (examples) {
\(\left\{
\begin{matrix}
({\color{bpurp}z_{chess}}, {\color{bpurp}z_{lose chess}})\\
$\vdots$
\end{matrix}\right\}\)};

\node[gray, text width=2cm, align=center] at (-4.1, -0.4) {Example network};
\node[block] at (-4.1, -1.25) (examplenet) {\(\mathcal{E}\)}; 
\path[arrow] (examples.east) -- ([xshift=-3]examplenet.west);

\node[borange, text width=2cm] at (-2, -1.25) (examplestaskrep) {\(z_{meta}\)}; 
\path[arrow] ([xshift=3]examplenet.east) -- (examplestaskrep.west);
}

\end{scope}
%% d) performing 
\draw[boundingbox, draw=gray, fill=white] (-1.5, -2.5) rectangle (8.5, 2.5);
\only<2->{
\node[borange] at (-0.75, 1.25) (taskrep) {\(z_{meta}\)};

\node[gray, text width=2cm, align=center] at (0.75, 2.1) {Hyper network};
\node[block] at (0.75, 1.25) (hypernet) {\(\mathcal{H}\)}; 
\path[arrow] (taskrep.east) -- ([xshift=-3]hypernet.west);
}

\node[bpurp] at (1.66, -1.25) (handrep) {\(z_{poker}\)};

\only<1>{
\node[bblue, block, dashed] at (3.5, -1.25) {??}; 
}
\uncover<1>{
\node at (4, -1.9) {\includegraphics[height=1cm]{figures/wizard.png}};
}
\uncover<2->{
\node[bblue, block, dashed] at (3.5, -1.25) (tasknet) {\(\mathcal{T}\)}; 
}
\path[arrow] (handrep.east) -- ([xshift=-3]tasknet.west);
\only<2->{
\node[bblue, text width=1.5cm, align=center] at (3.5, -2) {Task network};
\path[arrow, out=0, in=90] ([xshift=3]hypernet.east) to ([yshift=3]tasknet.north);
}

\node[bpurp] at (5.5, -1.25) (output) {\(z_{lose poker}\)};
\path[arrow] ([xshift=3]tasknet.east) -- (output.west);

%% d subpanel) performing basic from meta-mapped

\draw[boundingbox, draw=gray, fill=white] (4.25, 0) rectangle (8.5, 2.5);
\node[gray] at (6.35, 2.25) {Performing the new task};
\begin{scope}[scale=0.5, shift={(8.5,2.5)}, every node/.append style={transform shape}]

\node[gray, text width=2cm, align=center] at (2, 1.1) {Hyper network};
\node[block, semithick, rounded corners=2] at (2, 0.25) (hypernet2) {\(\mathcal{H}\)}; 

\node[text width=0.5cm] at (0.7, -1.25) (inputs2) {\includegraphics[width=0.5cm]{figures/2_of_spades.png}\\\includegraphics[width=0.5cm]{figures/4_of_hearts.png}};

\node[gray, text width=2cm, align=center] at (1.9, -2) {Perception network};
\node[block, semithick, rounded corners=2] at (1.9, -1.25) (perceptionnet2) {\(\mathcal{P}\)}; 
\path[arrow, semithick] (inputs2.east) -- ([xshift=-3]perceptionnet2.west);

\node[bgreen] at (3.2, -1.25) (handrep2) {\(z_{hand}\)};
\path[arrow, semithick] ([xshift=3]perceptionnet2.east) -- (handrep2.west);

\node[bblue, block, semithick, rounded corners=2, dash pattern=on 2pt off 2pt] at (4.5, -1.25) (tasknet2) {\(\mathcal{T}\)}; 
\node[bblue, text width=1.5cm, align=center] at (4.5, -2) {Task network};
\path[arrow, semithick] (handrep2.east) -- ([xshift=-3]tasknet2.west);
\path[arrow, semithick, out=0, in=90] ([xshift=3]hypernet2.east) to ([yshift=3]tasknet2.north);

\node[bgreen] at (5.65, -1.25) (betrep2) {\(z_{bet}\)};
\path[arrow, semithick] ([xshift=3]tasknet2.east) -- (betrep2.west);

\node[gray, text width=2cm, align=center] at (6.8, -2) {Action network};
\node[block, semithick, rounded corners=2] at (6.8, -1.25) (actionnet2) {\(\mathcal{A}\)}; 
\path[arrow, semithick] (betrep2.east) -- ([xshift=-3]actionnet2.west);

\node at (7.8, -1.25) (output2) {\bf \$};
\path[arrow] ([xshift=3]actionnet2.east) -- (output2.west);
\end{scope}

%% connecting to subpanel

\node[inner sep=0, outer sep=0] at (6.8, -0.8) (c1) {};
\node[inner sep=0, outer sep=0] at (4.5, -0.3) (c2) {};
\node[inner sep=0, outer sep=0] at (4, 0.5) (c3) {};
\path[draw, gray, very thick, out=0, in=-90, dash pattern=on 2pt off 2pt] (output.east) to (c1);
\path[draw, gray, very thick, out=90, in=0, dash pattern=on 2pt off 2pt] (c1) to (c2);
\path[draw, gray, very thick, out=180, in=-90, dash pattern=on 2pt off 2pt] (c2) to (c3);
\path[arrow, gray, very thick, out=90, in=180, dash pattern=on 2pt off 2pt] (c3) to ([xshift=-3]hypernet2.west);
}
\end{tikzpicture}}
\end{column}
\end{columns}
\end{frame}


\againframe<4>{basic_metamapping_analogy}

\againframe<3>{homm_arch}

\begin{frame}[standout]
We can use the same approaches and networks for both basic tasks and meta-mappings.
\end{frame}

%% TODO: reinsert?
%\begin{frame}{Interim summary}
%\begin{itemize}[<+->]
%\item Basic tasks are mappings from inputs to outputs (e.g. poker hands to bets), which can be inferred from examples.
%\item We represent the task-specific computations as relatively simple transformations, parameterized from a task embedding. The rest of input and output is shared across tasks.
%\item One type of flexibility is systematically altering these tasks (e.g. trying to bet on losing hands instead of winning ones).
%\item This can be seen as a \textbf{meta-mapping}, that is, a mapping which takes tasks as input and produces tasks as output (e.g. maps try-to-win-at-poker to try-to-lose-at-poker).
%\item We implement the meta-mapping function as a transformation of task embeddings.
%\item This mapping can be parsimoniously inferred and implemented using exactly the same networks as the basic tasks.
%\end{itemize}
%\end{frame}

\section{Explorations}

\begin{frame}{Explorations overview}
This same approach can be used in many different settings: 
\begin{table}
\center
\begin{tabular}{|c|c|c|}
\hline
Experiment & Type & Cognitive motivation \\
\hline
\color{lightgray} Polynomials & \color{lightgray} Regression & \color{lightgray} Proof of concept \\
Card games & Regression & Adapting strategies \\
2D worlds & RL & Adapting behaviors \\
Categories & Classification & Adapting concepts \\
\hline
\end{tabular}
\end{table}
\end{frame}

\againframe<5>{pokermotivation}

\begin{frame}{Simple card games}

\begin{itemize}
\item Constructed 5 two-card games (analogous to poker, blackjack, etc.)
\item With 8 variations of each, including ``try-to-lose'' variations
\item Taught the model non-losing variations of poker, and winning and losing variations of other games 
\item Can it infer losing at poker?
\end{itemize}
%\begin{columns}
%\begin{column}{0.5\textwidth}
%Five games:
%\begin{itemize}
%\item \textbf{High card:} Highest wins.
%\item \textbf{Pairs:} Pairs are valuable. 
%\item \textbf{Match:} Closest to matching cards win. 
%\item \textbf{Blackjack:} Value increases to a threshold, then switches to negative. 
%\item \textbf{Straight flush:} Adjacent same suit numbers are best, then adjacent different. 
%\end{itemize}
%\end{column}
%
%\begin{column}{0.5\textwidth}
%Three binary attributes:
%\begin{itemize}
%\item \textbf{Losers:} Try to lose instead of winning! Reverses the ranking of hands.
%\item \textbf{Suits rule:} Suits are more important than values. 
%\item \textbf{Switch suit:} Switches which of the suits is more valuable.
%\end{itemize}
%\vspace{3.2em}
%\end{column}
%\end{columns}
%\vspace{1em}
%\textbf{= 40 total, held out all four ``straight flush'' losing versions}
\end{frame}

\againframe<4>{homm_arch}

\begin{frame}
\frametitle<1>{Possible results}
\frametitle<2->{Meta-mapping results}
\only<1>{
\includegraphics[height=0.8\textheight]{figures/cards_patterns_interpretation.png}
}
\only<2>{
\includegraphics[height=0.8\textheight]{../../psych/dissertation/3-human-adaptation/figures/adaptation_HoMM_only.png}
\begin{tikzpicture}[overlay]
\fill[white] (-3, 2) rectangle (-0.2, 5); 
\end{tikzpicture}
}
\end{frame}

\begin{frame}[standout]
Meta-mapping adapts well! But how well?
\end{frame}

\againframe<5>{homm_arch}
%TODO: new language intro

\begin{frame}<1>[label=cards_results]
\frametitle<1>{MM vs. language}
\frametitle<2->{MM vs. humans}
\only<1>{ 
\includegraphics[height=0.8\textheight]{../../psych/dissertation/3-human-adaptation/figures/adaptation_HoMM_langauge.png}
\begin{tikzpicture}[overlay]
\fill[white] (-3, 3.5) rectangle (-0.2, 4); 
\end{tikzpicture}
}
\only<2>{
\includegraphics[height=0.8\textheight]{../../psych/dissertation/3-human-adaptation/figures/human_adaptation_vs_HoMM.png}
Note: Human variability is mostly internal, not due to experiment.
}
\only<3>{
\includegraphics[height=0.8\textheight]{../../psych/dissertation/3-human-adaptation/figures/change_scores.png}
Note: this analysis is biased against models because of ceiling.
}
\end{frame}

\begin{frame}{Human comparison}
\begin{columns}
\begin{column}{0.5\textwidth}
\begin{itemize}
\item On mTurk, taught 40 subjects this card game.
\item Instructions + hand comparisons to illustrate rules.
\item 32 practice trials where they saw outcome.
\item 24 testing trials without outcome.
\item Told to lose, 24 more testing trials without outcome.
\end{itemize}
\end{column}

\begin{column}{0.5\textwidth}
\vspace{1em}
\includegraphics[width=\textwidth]{../../psych/dissertation/3-human-adaptation/figures/pre_bet_screenshot.png}
\includegraphics[width=\textwidth]{../../psych/dissertation/3-human-adaptation/figures/post_bet_screenshot.png}
\end{column}
\end{columns}

\end{frame}

\againframe<2>{cards_results}

%\begin{frame}{Aside: Why are humans not optimal at winning?}
%\centering
%\includegraphics[height=0.8\textheight]{../../psych/dissertation/3-human-adaptation/figures/testing_basic_bet_densities.png}
%\end{frame}

\begin{frame}[standout]
On this task, MM adapts comparably to humans, and better than language alone. \\[1em]
\only<2->{
(I'm not saying that language isn't useful, just that transforming task representations may be useful as well.)
}
\end{frame}

\begin{frame}{RL tasks}
\begin{columns}
\begin{column}{0.5\textwidth}
Pick-up task:
\includegraphics[width=0.75\textwidth]{../../psych/dissertation/4-extending/figures/pick_up_0.png}%
\end{column}
\begin{column}{0.5\textwidth}
Pushing task:
\includegraphics[width=0.75\textwidth]{../../psych/dissertation/4-extending/figures/pusher_0.png}
\end{column}
\end{columns}
\vspace{1em}
{\bf
\(\mathbf{\times}\) 5 pairs of colors \(\mathbf{\times}\) binary switching of good and bad color \(\mathbf{=}\) 20 tasks.\\
Held out switched versions of two pairs in both task types, leaving 16 train tasks.
}
\end{frame}

\begin{frame}{RL tasks}
\inlineMovie{figures/recording_pickup.mp4}{../../psych/dissertation/4-extending/figures/pick_up_0.png}{width=0.48\textwidth, height=0.36\textwidth}%
\inlineMovie{figures/recording_pusher.mp4}{../../psych/dissertation/4-extending/figures/pusher_0.png}{width=0.48\textwidth, height=0.36\textwidth}
\end{frame}

\begin{frame}{RL tasks}
Why RL? 
\begin{itemize}
\item It has driven some substantial recent AI achievements.
\item It relates to neuroscience and cognition.
\item It requires more sophisticated adaptation. 
\end{itemize}
\end{frame}

%% Remember to say its okay not to know RL
%\begin{frame}<1>[label=extending_methods]
%\frametitle<1>{RL model changes}
%\frametitle<2>{Categorization model changes}
%\begin{columns}
%\begin{column}{0.5\textwidth}
%\only<2>{
%\begin{itemize}
%\item \(50 \times 50 \times 3\) RGB pixel input.
%\item Shared 4-layer CNN for input processing.  
%\item Cross-entropy loss.
%\end{itemize}
%}
%\only<1> {
%\begin{itemize}
%\item DQN-like approach.
%\item \(91 \times 91 \times 3\) RGB input.
%\item CNN for input, and Q-value (action-value) decoder shared across tasks. 
%\item Deeper task-specific net.
%\item Meta-network examples were (s, a, r) tuples.
%\item Task representations in training were partly persistent, to overcome strongly conflicting signals and accelerate learning.
%\end{itemize}
%}
%\end{column}
%\begin{column}{0.5\textwidth}
%\begin{tikzpicture}
%\draw[boundingbox, fill=white] (-3, -4.3) rectangle (2.5, 3.2);
%\node[lightgray] at (-1.2, 2.9) {Basic meta-learning};
%\node at (-1.5, -4) (examples) {Examples};
%\node at (-1.5, -3.2) (D1) {
%\(\left\{
%\begin{matrix}
%(x^{ex}_{in,0}, y^{ex}_{targ,0})\\
%$\vdots$
%\end{matrix}\right\}\)};
%
%\node at (1.5, -4) (probes) {Probes};
%\node at (1.5, -3.2) (D2) {
%%\(z^{prb}_{in}\)};
%\(\left\{
%\begin{matrix}
%x^{prb}_{in,0}\\
%$\vdots$
%\end{matrix}\right\}\)};
%
%\node [block] at (-1.5, -1.95) (M) {\(\mathcal{M}\)};
%\path [arrow] ([yshift=-5]D1.north) to (M);
%
%\node at (-1, -1) (zfunc) {\(z^{task}\)};
%\path [arrow, out=90, in=-135] (M.north) to ([xshift=6,yshift=3]zfunc.south west);
%
%\node[block] at (0, -0.55) (H) {\(\mathcal{H}\)};
%\path [arrow, out=45, in=180] ([yshift=-3]zfunc.north) to (H.west);
%\node [block, dashed] at (1.5, -0.55) (F) {\(F_{z^{task}}\)};
%\path[arrow] (H.east) to (F.west);
%\path [arrow, dashed] (D2) to (F);
%
%\node at (1.5, 1) (outputs) {
%%\(z^{pred}_{out}\)};
%\(\left\{
%\begin{matrix}
%y^{pred}_{out,0},\\
%$\vdots$
%\end{matrix}\right\}\)};
%\node at (1.5, 1.8) (predictions) {Predictions};
%
%\path [arrow, dashed] (F) to ([yshift=3]outputs.south);
%
%\node at (-1.5, 1.8) (probetargs) {Probe targets};
%\node at (-1.5, 1) (D2targs) {
%%\(z^{prb}_{in}\)};
%\(\left\{
%\begin{matrix}
%y^{prb}_{targ,0}\\
%$\vdots$
%\end{matrix}\right\}\)};
%
%\node [align=center, text width=1.25 cm] at (0, 2.25) (dispatch) {\baselineskip=12pt Loss\par};
%
%\path [arrow, dashed, out=180, in=-90] ([xshift=3]outputs.west) to (dispatch.south);
%
%\path [arrow, dashed, out=0, in=-90] ([xshift=-3.5]D2targs.east) to (dispatch.south);
%
%%% overlay
%
%\draw [fill=black, opacity=0.2] (0.75, -0.2) rectangle (2.25, 0.35);
%\draw [fill=red, opacity=0.4] (0.75, -1) rectangle (2.25, -0.2);
%\draw [fill=black, opacity=0.2] (0.75, -2.5) rectangle (2.25, -1);
%
%\uncover<4->{
%\draw [red, ultra thick] (-1, -1) circle (0.5);
%}
%\end{tikzpicture}
%\end{column}
%\end{columns}
%
%\end{frame}

\againframe<4>{homm_arch}

\begin{frame}{Results}
\includegraphics[height=0.8\textheight]{../../psych/dissertation/4-extending/figures/grids_adaptation_results.png}
\end{frame}

\begin{frame}[standout]
Meta-mapping can work in RL, even with a small support set of tasks. 
\end{frame}

\begin{frame}{Relation to model-based adaptation}
Model-based methods are often framed in part as an approach to flexibility.
\begin{itemize}
\item However, for this to work, these methods generally must have a new reward function handed to them.
\item That is, they're offloading a substantial part of the adaptation problem to another system.
\item Meta-mapping could offer a principled way to adapt a reward-predicting function, transition models, etc.
\item Thus meta-mapping and model-based planning could be complementary.
\end{itemize}
\end{frame}

%\begin{frame}{Behavioral uncertainty}
%\inlineMovie{figures/recording_pickup.mp4}{../../psych/dissertation/4-extending/figures/pick_up_0.png}{width=0.48\textwidth, height=0.36\textwidth}%
%\inlineMovie{figures/recording_pusher.mp4}{../../psych/dissertation/4-extending/figures/pusher_0.png}{width=0.48\textwidth, height=0.36\textwidth}
%\end{frame}
%

\begin{frame}{Categorization tasks}
\vspace{0.1em}
\begin{columns}
\begin{column}{\dimexpr\paperwidth-10pt}
\begin{tikzpicture}[auto]%, scale=0.8, every node/.style={scale=0.8}]
% blickets
\node at (-4.8, 2.2) {\LARGE ``Blickets''};
\draw[pen colour=fg, decoration={calligraphic brace, amplitude=0.5cm},decorate,line width=1mm] (-3, 0.3) -- (-3, 4); 
\node at (-2, 3.1) {\includegraphics[width=1.8cm]{../../psych/dissertation/4-extending/figures/categorization_stimuli/32_red_triangle_0.png}};
\node at (-2, 1.2) {\includegraphics[width=1.8cm]{../../psych/dissertation/4-extending/figures/categorization_stimuli/24_red_triangle_0.png}};

% not blickets
\node at (-4.3, -2.1) {\LARGE ``Not''};
\draw[pen colour=fg, decoration={calligraphic brace, amplitude=0.5cm},decorate,line width=1mm] (-3, -4) -- (-3, -0.3); 
\node at (-2, -3.1) {\includegraphics[width=1.8cm]{../../psych/dissertation/4-extending/figures/categorization_stimuli/32_red_circle_0.png}};
\node at (-2, -1.2) {\includegraphics[width=1.8cm]{../../psych/dissertation/4-extending/figures/categorization_stimuli/24_yellow_triangle_0.png}};
\only<2>{
\node at (1.3, 0) {\LARGE ``Blickets?''};
}
\uncover<3->{
\node at (3, 3.5) {\LARGE ``Zipfs = cyan blickets''};
\node at (1.6, 0) {\LARGE ``Zipfs?''};
}
\uncover<2->{
\draw[pen colour=fg, decoration={calligraphic brace, amplitude=0.5cm},decorate,line width=1mm] (3.2, -2.9) -- (3.2, 2.9); 
\node at (4.2, 1.9) {\includegraphics[width=1.8cm]{../../psych/dissertation/4-extending/figures/categorization_stimuli/24_red_triangle_1.png}};
\node at (4.2, 0) {\includegraphics[width=1.8cm]{../../psych/dissertation/4-extending/figures/categorization_stimuli/32_cyan_triangle_1.png}};
\node at (4.2, -1.9) {\includegraphics[width=1.8cm]{../../psych/dissertation/4-extending/figures/categorization_stimuli/32_cyan_inverseplus_1.png}};
}

\end{tikzpicture}
\end{column}
\end{columns}
\end{frame}


\begin{frame}{Why categorization?}
\begin{itemize}
\item Cognitive history.
\item Rich visual input, more complex tasks.
\item Classification is a common problem.
\end{itemize}
\end{frame}
\begin{frame}{Categorization tasks}
\begin{columns}
\begin{column}{0.5\textwidth}
Types of categories:
\begin{itemize}
\item Basic tasks: single attribute (shape/color/size) category.
\item Composite tasks: AND, OR, or XOR of two different attributes.  
\end{itemize}
\uncover<2->{
Meta-mappings:
\begin{itemize}
\item Switch one color to another or switch one shape to another. 
\item (Omitted size switching because only a few sizes.)
\end{itemize}
}
\end{column}
\begin{column}{0.5\textwidth}
\centering
\includegraphics[width=2.2cm]{../../psych/dissertation/4-extending/figures/categorization_stimuli/24_red_triangle_1.png}\\
\includegraphics[width=2.2cm]{figures/categorization/32_cyan_emptysquare_0.png}\\
\includegraphics[width=2.2cm]{../../psych/dissertation/4-extending/figures/categorization_stimuli/24_yellow_inverseplus_0.png}
\end{column}
\end{columns}
\end{frame}

\againframe<6>{homm_arch}
%\againframe<2>{extending_methods}

\begin{frame}<1>[label=cat_results]
\frametitle{Categorization results}
\only<1>{
\includegraphics[height=0.8\textheight]{figures/categories_adaptation_no_lh.png}
}
\only<2>{
\includegraphics[height=0.8\textheight]{figures/categories_adaptation.png}
}
\end{frame}

%\againframe<4>{homm_arch}
\againframe<2>{cat_results}

\begin{frame}[standout]
%Language produces better concept representations than examples.\\[1em]
%Meta-mapping these representations produces similar performance to language generalization.
\end{frame}

%\begin{frame}{Thoughts on language \& my recent RL generalization work}
%How does this relate to our recent finding that realistic tasks improve generalization in language-conditioned RL?\par
%\begin{itemize}[<+(1)->]
%\item In some cases meta-mapping appears to be quite sample efficient, e.g. only 16 training tasks in RL.
%\item This is despite having \textbf{more challenging} evaluation than the prior language work, since often the adapted task \textbf{contradicts previous learning} (e.g. was negatively rewarded). 
%\item But language constructs better task representations in the categorization domain, with language + meta-mapping adapting similarly to language generalization. 
%\item Whether meta-mapping is computationally useful with data as rich as human experience is an open question.
%\end{itemize}
%(``Environmental drivers of generalization in a situated agent,'' Hill, Lampinen, et al., ICLR 2020)
%\end{frame}

\section{Why zero-shot adaptation?}

\begin{frame}{Zero-shot adaptation is a useful starting point}
Because it accelerates learning and reduces cumulative mistakes.
{
\centering
\only<1>{
\includegraphics[height=0.8\textheight]{../../psych/dissertation/5-timescales/figures/polynomial_optimization_curves.png}
}
\only<2>{
\includegraphics[height=0.8\textheight]{../../psych/dissertation/5-timescales/figures/polynomial_optimization_cumulative_regret.png}
}
}
\end{frame}

\begin{frame}[standout]
Starting from a zero-shot ``guess'' at the appropriate behavior results in faster learning, and reduces cumulative mistakes by an order of magnitude 
\end{frame}


%\section{Wrapping up}

\begin{frame}{Conclusions}

\begin{itemize}
\item Humans can perform novel tasks zero-shot. 
\item We suggest a computational perspective on this: \textbf{meta-mappings}, which map a task to a transformed version of that task. 
\item We propose a parsimonious \textbf{HoMM} architecture that: 
    \begin{itemize}
    \item Performs basic tasks by parameterizing a small task network from a vector representation of the task. 
    \item Uses an analogy between basic tasks and meta-mappings to perform meta-mappings by transforming task embeddings. 
    \end{itemize}
\item This approach performs comparably to humans in a card game task, and outperforms (or equals) a language based approach across three settings.
\item Zero-shot adaptation allows faster, better later learning. 
\item Earlier paper: \url{https://arxiv.org/abs/1905.09950} and library + all code: \url{https://github.com/lampinen/HoMM}
\end{itemize}
\end{frame}

%\section{A brief methods comment}
%\begin{frame}{Why you should implement everything twice}
%\only<1>{
%Bugs that slipped past my unit testing, but that I found by reimplementing my code in a shared library, and comparing results:%
%\begin{itemize}
%\item Off-by-one error in array indexing leading to unintentional, weird weight tying in hyper network output. 
%\item Target network for DQN accidentally having one set of weights shared with main net all the time.
%\item Using mixed task representations for evaluation rather than cached ones in RL tasks.
%\item Error in domains of meta-mappings in categorization tasks.
%\item Implementation of suit \(\leftrightarrow\) rank in card tasks not as intended.
%\end{itemize}
%None of these made my earlier results invalid, but they did worsen performance in some cases, and/or made it so I wasn't really doing what I thought/said I was. All results I showed you today are from corrected code.
%}
%\only<2>{
%But I was lucky:\\
%\includegraphics[width=\textwidth]{figures/correction.png}
%}
%\end{frame}

\begin{frame}[standout]
%Thanks to Jay, Noah, Surya, Erin, Katherine, Arianna and the rest of the lab! \\[1em]
%And you for listening! \\[1em]
Questions?
\end{frame}

\end{document}
