\documentclass{beamer}
\setbeamerfont{subsection in toc}{size=\footnotesize}
%\setbeameroption{show notes on second screen=left} %enable for notes
\usepackage{graphicx}
\usepackage{xcolor}
\usepackage{listings}
\usepackage{hyperref}
\lstset{language=python,frame=single}
\usepackage{verbatim}
\usepackage[longnamesfirst]{natbib}
\usepackage{subcaption}
\usepackage{amsmath}
\usepackage{bm}
\usepackage{relsize}
\usepackage{appendixnumberbeamer}
\usepackage{xparse}
\usepackage{multimedia}
\usepackage{xcolor}
\usepackage[normalem]{ulem}
\usepackage{hyperref}
\usepackage{tikz}
\usetikzlibrary{matrix,backgrounds}
\usetikzlibrary{positioning}
\usetikzlibrary{shapes,arrows}
\usetikzlibrary{positioning}

\captionsetup[subfigure]{labelformat=empty}

\tikzset{onslide/.code args={<#1>#2}{%
  \only<#1>{\pgfkeysalso{#2}} 
}}

\tikzstyle{block} = [rectangle, draw, thick, align=center, rounded corners]
\tikzstyle{boundingbox} = [thick, lightgray]
\tikzstyle{dashblock} = [rectangle, draw, thick, align=center, dashed]
\tikzstyle{conc} = [ellipse, draw, thick, dashed, align=center]
\tikzstyle{netnode} = [circle, draw, very thick, inner sep=0pt, minimum size=0.5cm]
\tikzstyle{relunode} = [rectangle, draw, very thick, inner sep=0pt, minimum size=0.5cm]
\tikzstyle{line} = [draw, very thick, -latex']
\tikzstyle{arrow} = [draw, ->, very thick]

\definecolor{bpurp}{HTML}{984ea3}
\definecolor{bblue}{HTML}{377eb8}

\usetheme[numbering=none]{metropolis}

\begin{document}

\title{Embodiment and environmental realism lead to more systematic generalization}
\author{Felix Hill, Andrew Lampinen, Rosalia Schneider, Stephen Clark, Matthew Botvinick, James L. McClelland, Adam Santoro}
\date{}
\frame{\titlepage}

\begin{frame}{Deep reinforcement learning systems can do cool things!}
\vspace{0.5em}
\centering
\includegraphics[height=0.4\paperheight]{figures/breakout.png}\\[-2pt]
\includegraphics[height=0.4\paperheight]{figures/move37.jpeg}%
\includegraphics[height=0.4\paperheight]{figures/alphastar.jpeg}
\end{frame}

\begin{frame}[standout]
But when do they generalize what they learned? 
\end{frame}

\begin{frame}{An important question for deep RL as a tool or cognitive model}
\centering
\includegraphics[width=\textwidth]{figures/deep_RL_as_a.png}
\end{frame}

\begin{frame}{Systematic compositional generalization in language}
\vspace{-2em}
\centering
\includegraphics[width=\textwidth]{figures/fodor_compositionality_1.png}
\end{frame}

\begin{frame}{Fodor on neural nets (not verbatim)}
\centering
\includegraphics[width=0.75\textwidth]{figures/fodor_zoidberg.png}
\end{frame}


\begin{frame}{But we know neural nets can be systematic in theory}
\vspace{5em}
{
\centering
\includegraphics[width=\textwidth]{figures/universal_approximator.png}
}
\vspace{3em}

{\small (Siegelman \& Sontag, 1992)}
\end{frame}

\begin{frame}[standout]
So the question is, in practice, when do deep networks learn to generalize systematically? 
\end{frame}

\begin{frame}{Color-shape composition experiment}
\centering
\only<1>{
\includegraphics[width=0.8\textwidth]{figures/color_shape_composition.png}
}
\only<2>{
\includegraphics[width=0.8\textwidth]{figures/color_shape_design.png}
}
\end{frame}

\begin{frame}{Comparing different versions of the same abstract task}
\centering
\includegraphics[width=\textwidth]{figures/different_versions_1.png}
\end{frame}

\begin{frame}{Behaving over time results in better generalization!}

\end{frame}

\begin{frame}[standout]
Generalization is better in a richer environment where the agent acts over time.
\end{frame}

\begin{frame}{Verb-noun composition experiment: 2D vs. 3D}

\end{frame}

\begin{frame}[standout]
Generalization is better in the 3D environment.
\end{frame}

\begin{frame}{Egocentric perspective}

\end{frame}

\begin{frame}[standout]
Generalization in the 2D environment with an egocentric perspective is closer to 3D!
\end{frame}

\begin{frame}{What does this mean?}
\begin{itemize}
\item It's not just the abstract task that matters for generalization, but the specifics of the setting in which it is instantiated. 
\item Embodiment may improve generalization -- performance is better when the agent acts rather than classifying, and when it has an egocentric perspective.
\item It's not obvious that egocentric perspective should help with verb-noun recomposition -- the agent is solving the navigation problem perfectly on the trained ``put'' tasks.
\item We don't have a theory for what features affect generalization.
\item So we shouldn't dismiss deep RL as a cognitive model just because it fails in unrealistic tasks. 
\end{itemize}
\end{frame}

\begin{frame}[standout]
Thanks!\\
\url{https://arxiv.org/abs/1910.00571}
\end{frame}


\end{document}
