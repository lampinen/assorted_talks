\documentclass{beamer}
\setbeamerfont{subsection in toc}{size=\footnotesize}
%\setbeameroption{show notes on second screen=left} %enable for notes
\usepackage{graphicx}
\usepackage{xcolor}
\usepackage{listings}
\usepackage{hyperref}
\lstset{language=python,frame=single}
\usepackage{verbatim}
\usepackage[longnamesfirst]{natbib}
\usepackage{subcaption}
\usepackage{amsmath}
\usepackage{bm}
\usepackage{relsize}
\usepackage{appendixnumberbeamer}
\usepackage{xparse}
\usepackage{multimedia}
\usepackage{xcolor}
\usepackage[normalem]{ulem}
\usepackage{hyperref}
\usepackage{tikz}
\usetikzlibrary{matrix,backgrounds}
\usetikzlibrary{positioning}
\usetikzlibrary{shapes,arrows}
\usetikzlibrary{positioning}

\captionsetup[subfigure]{labelformat=empty}

\tikzset{onslide/.code args={<#1>#2}{%
  \only<#1>{\pgfkeysalso{#2}} 
}}

\tikzstyle{block} = [rectangle, draw, thick, align=center, rounded corners]
\tikzstyle{boundingbox} = [thick, lightgray]
\tikzstyle{dashblock} = [rectangle, draw, thick, align=center, dashed]
\tikzstyle{conc} = [ellipse, draw, thick, dashed, align=center]
\tikzstyle{netnode} = [circle, draw, very thick, inner sep=0pt, minimum size=0.5cm]
\tikzstyle{relunode} = [rectangle, draw, very thick, inner sep=0pt, minimum size=0.5cm]
\tikzstyle{line} = [draw, very thick, -latex']
\tikzstyle{arrow} = [draw, ->, very thick]

\definecolor{bpurp}{HTML}{984ea3}
\definecolor{bblue}{HTML}{377eb8}

\usetheme[numbering=none]{metropolis}

\begin{document}

\title{Embodiment and environmental realism lead to more systematic generalization}
\author{Felix Hill, Andrew Lampinen, Rosalia Schneider, Stephen Clark, Matthew Botvinick, James L. McClelland, Adam Santoro}
\date{}
\frame{\titlepage}

\begin{frame}[standout]
Many slides lifted from my upcoming lightning talk -- so may be a little cartoony.
\end{frame}

\begin{frame}{Deep reinforcement learning systems can do cool things!}
\vspace{0.5em}
\centering
\includegraphics[height=0.4\paperheight]{figures/breakout.png}\\[-2pt]
\includegraphics[height=0.4\paperheight]{figures/move37.jpeg}%
\includegraphics[height=0.4\paperheight]{figures/alphastar.jpeg}
\end{frame}

\begin{frame}[standout]
But when do they generalize what they learned? 
\end{frame}

\begin{frame}{An important question for deep RL as a tool or cognitive model}
\centering
\includegraphics[width=\textwidth]{figures/deep_RL_as_a.png}
\end{frame}

\begin{frame}{Systematic compositional generalization in language}
\vspace{-2em}
\centering
\includegraphics[width=\textwidth]{figures/fodor_compositionality_1.png}
\end{frame}

\begin{frame}{Fodor on neural nets (not verbatim)}
\centering
\includegraphics[width=0.75\textwidth]{figures/fodor_zoidberg.png}
\end{frame}


\begin{frame}{But we know neural nets can be systematic in theory}
\vspace{5em}
{
\centering
\includegraphics[width=\textwidth]{figures/universal_approximator.png}
}
\vspace{3em}

{\small (Siegelman \& Sontag, 1992)}
\end{frame}

\begin{frame}[standout]
So the question is, in practice, when do deep networks learn to generalize systematically? 
\end{frame}

\section{Demonstrating systematic compositional generalization}

\begin{frame}{Agent sketch}
\begin{itemize}
\item 3-layer ResNet. 
\item LSTM core. 
\item Policy + value output heads, each a 2(?) layer network.
\item Trained with IMPALA.
\item (Basically the same for the next talk too.)
\end{itemize}
\end{frame}

\begin{frame}{Verb-noun composition experiment}
\centering
\includegraphics[width=0.8\textwidth]{figures/putting_design.png}
\end{frame}

\begin{frame}[standout]
\url{https://www.youtube.com/embed/iuihBGAGI3M}
\end{frame}

\begin{frame}{Pretty systematic!}
\centering
\includegraphics[width=0.8\textwidth]{figures/plots/putting_generalization.png}
\end{frame}

\begin{frame}[standout]
But what are the drivers of this systematic generalization?
\end{frame}

\section{Investigating the features underlying systematicity}

\begin{frame}{Color-shape composition experiment}
\centering
\only<1>{
\includegraphics[width=0.8\textwidth]{figures/color_shape_composition.png}
}
\only<2>{
\includegraphics[width=0.8\textwidth]{figures/color_shape_design.png}
}
\end{frame}

\begin{frame}{Comparing different versions of the same abstract task}
\centering
\includegraphics[width=\textwidth]{figures/different_versions_1.png}
\end{frame}

\begin{frame}{Behaving over time results in better generalization!}
\includegraphics[width=\textwidth]{figures/plots/agent_vs_classifier.png}
{\small Error-bars throughout are \(\pm\)SEM}
\end{frame}

\begin{frame}{Why?}
\begin{itemize}
\item There's essentially an implicit form of data augmentation for the agent -- it gets to see each training object from more perspectives as it moves around.
\item This also means it gets more perspectives on each \textbf{testing} object, and to integrate information across them.
\item And it has more chances to correct an error, since it's making a sequence of decisions before the episode ends, rather than a single classification. 
\end{itemize}
\end{frame}

\begin{frame}[standout]
Generalization is better in a richer environment where the agent acts over time.
\end{frame}

\begin{frame}{Verb-noun composition experiment (II)}
\begin{center}
\includegraphics[width=0.8\textwidth]{figures/2D_3D_design.png}
\end{center}
Note that there are fewer objects than before because of limitations of one instantiation.
\end{frame}

\begin{frame}{Comparing different versions of the same abstract task}
\centering
\includegraphics[width=\textwidth]{figures/different_versions_2.png}
\end{frame}

\begin{frame}{Verb-noun composition experiment: 2D vs. 3D}
\includegraphics[width=\textwidth]{figures/plots/2D_vs_3D.png}
{\small Note: 3D generalization is worse than before because fewer objects.}
\end{frame}

\begin{frame}[standout]
Generalization is better in the 3D environment. Why?
\end{frame}

\begin{frame}{Frames of reference}
\only<1>{
The first-person perspective of the 3D agent adds a certain amount of invariance, so we compared two frames of reference in 2D. 
\vspace{2em}
\begin{columns}
\begin{column}{0.45\textwidth}
\textbf{Allocentric:}
    \begin{itemize}
    \item World fixed.
    \item Agent moves within it.
    \item This is how most grid-worlds work, including last slide's. 
    \end{itemize}
\end{column}
\begin{column}{0.45\textwidth}
\textbf{Egocentric:}
    \begin{itemize}
    \item Agent fixed at center.
    \item World moves around it.
    \item More like human fixation and/or first-person perspective of 3D.
    \end{itemize}
\end{column}
\end{columns}
}
\only<2>{
\includegraphics[width=\textwidth]{figures/allo_vs_ego/0.png}
}
\only<3>{
\includegraphics[width=\textwidth]{figures/allo_vs_ego/1.png}
}
\only<4>{
\includegraphics[width=\textwidth]{figures/allo_vs_ego/2.png}
}
\only<5>{
\includegraphics[width=\textwidth]{figures/allo_vs_ego/3.png}
}
\end{frame}

\begin{frame}{Egocentric perspective helps}
\vspace{1em}
\centering
\includegraphics[width=\textwidth]{figures/plots/2D_vs_scrolling_vs_3D.png}
\end{frame}

\begin{frame}[standout]
Generalization in the 2D environment with an egocentric perspective is closer to 3D!
\end{frame}

\begin{frame}{Controls}
We also ran some controls:
\begin{itemize}
\item Reducing the agent recurrent hidden size to regularize in 2D does not seem to help, so it's not as simple as 2D requires less capacity. 
\item It does not appear to be about partial observability in the egocentric case.
\item Making the agent visible behind the object its carrying in 2D actually makes performance \textbf{worse} in both scrolling and non-scrolling condiitions.
\end{itemize}
(Note: not all of these are reported in the paper.)
\end{frame}

\begin{frame}{Negation experiment (training set size)}
\begin{itemize}
\item Train to find 8 + N objects.
\item Trained on negation with the N objects.
\item Eval on the held-out 8.
\item How does train set size N affect systematicity?
\item Not especially surprising, but its worth highlighting how systematicity increases, because it's hard to memorize so much.
\end{itemize}
\end{frame}

\begin{frame}{Negation}
\vspace{1em}
\centering
\includegraphics[width=\textwidth]{figures/plots/negation.png}
\end{frame}

\begin{frame}[standout]
Unsurprisingly, generalization is more systematic with a bigger training set size.
\end{frame}

\begin{frame}{Conclusions}
\begin{itemize}
\item It's not just the abstract task that matters for generalization, but the specifics of the setting in which it is instantiated. 
\item Embodiment may improve generalization -- performance is better when the agent acts rather than classifying a frame, and when it has an egocentric perspective.
\item It's not obvious that egocentric perspective should help with verb-noun recomposition -- the agent is solving the navigation problem perfectly on the trained ``put'' tasks either way.
\item We don't have a complete theory for what features affect generalization.
\item So we shouldn't dismiss deep RL's potential to generalize systematically just because it fails in unrealistic tasks. 
\end{itemize}
\end{frame}

\begin{frame}[standout]
Thanks!\\
\url{https://arxiv.org/abs/1910.00571}
\end{frame}


\end{document}
